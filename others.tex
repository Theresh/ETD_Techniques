\documentclass[journal,12pt,twocolumn]{IEEEtran}
%
\makeatletter
\makeatother
\usepackage{setspace}
\usepackage{gensymb}
\usepackage{xcolor}
\usepackage{caption}
%\usepackage{stackengine}
%\usepackage{subcaption}
%\doublespacing
\singlespacing



\usepackage{graphicx}
\graphicspath{ {./images}  }
%\usepackage{amssymb}
%\usepackage{relsize}
\usepackage[cmex10]{amsmath}
\usepackage{mathtools}
%\usepackage{amsthm}
\interdisplaylinepenalty=2500
%\savesymbol{iint}
%\usepackage{txfonts}
%\restoresymbol{TXF}{iint}
\usepackage{wasysym}
\usepackage{amsthm}
\usepackage{mathrsfs}
\usepackage{txfonts}
\usepackage{stfloats}
\usepackage{cite}
\usepackage{cases}
\usepackage{mathtools}
\usepackage{subfig}
\usepackage{enumerate}	
\usepackage{enumitem}
\usepackage{amsmath}
%\usepackage{xtab}
\usepackage{longtable}
\usepackage{multirow}
%\usepackage{algorithm}
%\usepackage{algpseudocode}
\usepackage{enumitem}
\usepackage{mathtools}
%\usepackage{iithtlc}
%\usepackage[framemethod=tikz]{mdframed}
\usepackage{listings}
\usepackage{listings}
    \usepackage[latin1]{inputenc}                                 %%
    \usepackage{color}                                            %%
    \usepackage{array}                                            %%
    \usepackage{longtable}                                        %%
    \usepackage{calc}                                             %%
    \usepackage{multirow}                                         %%
    \usepackage{hhline}                                           %%
    \usepackage{ifthen}                                           %%
  %optionally (for landscape tables embedded in another document): %%
    \usepackage{lscape}     



%\usepackage{stmaryrd}


%\usepackage{wasysym}
%\newcounter{MYtempeqncnt}
\DeclareMathOperator*{\Res}{Res}
%\renewcommand{\baselinestretch}{4}
%\setcounter{secnumdepth}{4}
\renewcommand\thesection{\arabic{section}}
\renewcommand\thesubsection{\thesection.\arabic{subsection}}
\renewcommand\thesubsubsection{\thesubsection.\arabic{subsubsection}}
%\renewcommand\thesubsubsubsection{\thesubsubsection.\arabic{subsubsubsection}}

%\renewcommand\thesectiondis{\arabic{section}}
%\renewcommand\thesubsectiondis{\thesectiondis.\arabic{subsection}}
%\renewcommand\thesubsubsectiondis{\thesubsectiondis.\arabic{subsubsection}}
%\renewcommand\thesubsubsubsectiondis{\thesubsubsectiondis.\arabic{subsubsubsection}}
% correct bad hyphenation here
\hyphenation{op-tical net-works semi-conduc-tor}

%\lstset{
%language=C,
%frame=single, 
%breaklines=true
%}

%\lstset{
	%%basicstyle=\small\ttfamily\bfseries,
	%%numberstyle=\small\ttfamily,
	%language=Octave,
	%backgroundcolor=\color{white},
	%%frame=single,
	%%keywordstyle=\bfseries,
	%%breaklines=true,
	%%showstringspaces=false,
	%%xleftmargin=-10mm,
	%%aboveskip=-1mm,
	%%belowskip=0mm
%}

%\surroundwithmdframed[width=\columnwidth]{lstlisting}
\def\inputGnumericTable{}                                 %%

\lstset{
%language=python,
frame=single, 
breaklines=true,
columns=fullflexible
}

 

\begin{document}
%

\theoremstyle{definition}
\newtheorem{theorem}{Theorem}[section]
\newtheorem{problem}{Problem}
\newtheorem{proposition}{Proposition}[section]
\newtheorem{lemma}{Lemma}[section]
\newtheorem{corollary}[theorem]{Corollary}
\newtheorem{example}{Example}[section]
\newtheorem{definition}{Definition}[section]
%\newtheorem{algorithm}{Algorithm}[section]
%\newtheorem{cor}{Corollary}
\newcommand{\BEQA}{\begin{eqnarray}}
\newcommand{\EEQA}{\end{eqnarray}}
\newcommand{\define}{\stackrel{\triangle}{=}}

\bibliographystyle{IEEEtran}
%\bibliographystyle{ieeetr}

\providecommand{\nCr}[2]{\,^{#1}C_{#2}} % nCr
\providecommand{\nPr}[2]{\,^{#1}P_{#2}} % nPr
\providecommand{\mbf}{\mathbf}
\providecommand{\pr}[1]{\ensuremath{\Pr\left(#1\right)}}
\providecommand{\qfunc}[1]{\ensuremath{Q\left(#1\right)}}
\providecommand{\sbrak}[1]{\ensuremath{{}\left[#1\right]}}
\providecommand{\lsbrak}[1]{\ensuremath{{}\left[#1\right.}}
\providecommand{\rsbrak}[1]{\ensuremath{{}\left.#1\right]}}
\providecommand{\brak}[1]{\ensuremath{\left(#1\right)}}
\providecommand{\lbrak}[1]{\ensuremath{\left(#1\right.}}
\providecommand{\rbrak}[1]{\ensuremath{\left.#1\right)}}
\providecommand{\cbrak}[1]{\ensuremath{\left\{#1\right\}}}
\providecommand{\lcbrak}[1]{\ensuremath{\left\{#1\right.}}
\providecommand{\rcbrak}[1]{\ensuremath{\left.#1\right\}}}
\theoremstyle{remark}
\newtheorem{rem}{Remark}
\newcommand{\sgn}{\mathop{\mathrm{sgn}}}
\providecommand{\abs}[1]{\left\vert#1\right\vert}
\providecommand{\res}[1]{\Res\displaylimits_{#1}} 
\providecommand{\norm}[1]{\lVert#1\rVert}
\providecommand{\mtx}[1]{\mathbf{#1}}
\providecommand{\mean}[1]{E\left[ #1 \right]}
\providecommand{\fourier}{\overset{\mathcal{F}}{ \rightleftharpoons}}
%\providecommand{\hilbert}{\overset{\mathcal{H}}{ \rightleftharpoons}}
\providecommand{\system}{\overset{\mathcal{H}}{ \longleftrightarrow}}
	%\newcommand{\solution}[2]{\textbf{Solution:}{#1}}
\newcommand{\solution}{\noindent \textbf{Solution: }}
\providecommand{\dec}[2]{\ensuremath{\overset{#1}{\underset{#2}{\gtrless}}}}
\DeclarePairedDelimiter{\ceil}{\lceil}{\rceil}
%\numberwithin{equation}{subsection}
\numberwithin{equation}{section}
%\numberwithin{problem}{subsection}
%\numberwithin{definition}{subsection}
%\makeatletter
%\@addtoreset{figure}{section}
%\makeatother

\let\StandardTheFigure\thefigure
%\renewcommand{\thefigure}{\theproblem.\arabic{figure}}
%\renewcommand{\thefigure}{\thesection}


%\numberwithin{figure}{subsection}

%\numberwithin{equation}{subsection}
%\numberwithin{equation}{section}
%\numberwithin{equation}{problem}
%\numberwithin{problem}{subsection}
%\numberwithin{problem}{section}
%%\numberwithin{definition}{subsection}
%\makeatletter
%\@addtoreset{figure}{problem}
%\makeatother
%\makeatletter
%\@addtoreset{table}{problem}
%\makeatother

\let\StandardTheFigure\thefigure
\let\StandardTheTable\thetable
%%\renewcommand{\thefigure}{\theproblem.\arabic{figure}}
%\renewcommand{\thefigure}{\theproblem}

%%\numberwithin{figure}{section}

%%\numberwithin{figure}{subsection}



\def\putbox#1#2#3{\makebox[0in][l]{\makebox[#1][l]{}\raisebox{\baselineskip}[0in][0in]{\raisebox{#2}[0in][0in]{#3}}}}
     \def\rightbox#1{\makebox[0in][r]{#1}}
     \def\centbox#1{\makebox[0in]{#1}}
     \def\topbox#1{\raisebox{-\baselineskip}[0in][0in]{#1}}
     \def\midbox#1{\raisebox{-0.5\baselineskip}[0in][0in]{#1}}




\title{ 
%	\logo{
Efficient Transmitter Design Techniques in Digital Communication
%	}
}



\author{Theresh Babu Benguluri,Sandeep Kumar,Siddharth Maurya,Sai Manasa,Raktim Goswami,Abhishek Bairagi , G V V Sharma$^{*}$% <-this % stops a space
\thanks{*The author is with the Department
of Electrical Engineering, Indian Institute of Technology, Hyderabad
502285 India e-mail:  gadepall@iith.ac.in.}
}


% make the title area
\maketitle

\tableofcontents

%\bigskip
%
\begin{abstract}
\boldmath
A brief description about the Efficient Transmitter Design techniques. Which includes Interleaver/Deinterleaver for combating bursty errors, Physical Layer Framing for the efficient detection of Frame starting, and  Pulse Shaping for combating the InterSymbol Interference.
\end{abstract}

%\IEEEpeerreviewmaketitle

\section{Interleaver/Deinterleaver}


For 8PSK, 16APSK, and 32APSK modulation formats, the output of the LDPC encoder shall be bit interleaved using a
block interleaver. Data is serially written into the interleaver column-wise, and serially read out row-wise (the MSB of BBHEADER is read out first, except 8PSK rate 3/5 case where MSB of BBHEADER is read out third).\\
\begin{figure}
\begin{center}
\includegraphics[width=\columnwidth]{./figs/Interleaver.eps}
\end{center}
\caption{Bit interleaver for 8PSK}
\label{fig:inter}
\end{figure}

Fig. \ref{fig:inter} shows the comparison of 8PSK mapping schem with interleaver and without interleaver.

The configuration of the block interleaver for each modulation format is specified \cite{dvb}.





\section{Physical Layer Framing(PLFRAMING)}
\begin{itemize}
\item In order to form the PLFRAME which is used for receiver synchronization, a header is inserted in the beginning of each frame.



\begin{figure}
\begin{center}
\includegraphics[width=\columnwidth]{/pl_framing}
\end{center}
\caption{Structure of PLFRAME.}
\label{fig:splframe}
\end{figure}

 
\item If pilots are used, then a 36 symbols
pilot field is inserted every 16 slots of data symbols. Given that a PLFRAME cannot terminate to a pilot field, the total number of pilot fields is $\alpha_{PIL}=\left\lfloor \frac{(S-1)}{16} \right\rfloor$
\item The total length of the PLFRAME in symbols is 
\begin{equation}
K = 
\begin{cases}
90\times (S+1)  & \textit{with out pilots}
\\
90\times (S+1) + 36\times \alpha_{PIL}  & \textit{with pilots}
\end{cases}
\end{equation}
%\begin{center}
%\includegraphics[scale=0.35]{/parameters_pl_frame}
%\end{center} 

\begin{figure}
\begin{center}
\includegraphics[width=\columnwidth]{//parameters_pl_frame}
\end{center}
\caption{paramters of plframe}
\label{fig:plframe}
\end{figure}

\item The different frame formats,parameters of the normal andshort DVB-S2 frame for all constellation formats were tabulated in above table as  specified in \cite{dvb}
\end{itemize}

\section{Pulse Shaping}
\begin{itemize}
\item After randomization, the signals shall be square root raised cosine filtered. The roll-off factor shall be $\alpha$ = 0,35, 0,25and 0,20, depending on the service requirements.
\item Suitable $H(f)$ will be choosen from the \cite{dvb}.
\begin{align}
H(f) =\begin{cases}
1& \abs{f}<f_N(1-\alpha)\\
\cbrak{\frac{1}{2}+\frac{1}{2}\sin \frac{\pi}{2}\sbrak{\frac{f_N-\abs{f}}{\alpha}}}^{\frac{1}{2}}&\abs{f}=f_N(1-\alpha)\\
0&\abs{f}>f_N(1-\alpha)
\end{cases}
\end{align}

\end{itemize}

\bibliography{IEEEabrv,others.bib}
\end{document}

